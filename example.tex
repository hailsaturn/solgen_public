#colnames
0: surname
1: firstName
2: studentID
3: email
4: q1
5: q2
6: q3	
7: q4
8: q5

#identifier
2

#writepath
Output

#csv 
inputs.csv

#evaluation
prod1: q1_p * q1_q
prod2: prod1 + q1_c
formula: q1_p + x*q1_q + q1_c*x^2
legs: q2_animal == "cat" => 4 else 2
q3_elements: [5,6,7,8]
q3_choice: q3_elements[q3_index]
q3_cd: q3_c*q3_d
q3_ce: q3_c*q3_e
q3_cde: q3_c*q3_d + q3_e
q3_cde2: q3_c*q3_d + q3_e + 0.5
q3_cde2_round: round(q3_cde2,3)
f : x -> x^2
g : x -> 2*x/7
h : x,y -> x+y
flist: [f,g,h]
fevals: [f,g,h](3)
fe: fevals[0]
ge: fevals[1]
he: fevals[2]
he2: h(1,2)
q3c1: q3_c(1)
q3c12: q3_c(1,2)
q3cev: q3_c()

\documentclass[a4paper,12pt]{article}
% When the string "\documentclass" is detected, the solgen importer moves to "LaTeX mode", importing the file as LaTeX text and applying substitutions as needed.
% You can use any packages you like. It's treated as a standard LaTeX file

\usepackage{amsmath,amssymb,geometry}
\geometry{tmargin=1.2cm,bmargin=1.7cm,lmargin=1.2cm,rmargin=1.7cm}


\begin{document}

{
\Large
\noindent 
\bfseries
Text generated for @firstName @surname (@studentID)
}

\bigskip
\noindent

\begin{enumerate}
	\item
		Note: the number specified under the \#identifier banner is the column number used to uniquely identify each row.
		It is used to name the resulting LaTeX files.
	\item 
		Column E (index 4) has colname q1, so its variables are prefixed with ``q4''.

		Use variables in the template by prefixing them with an @ sign.
		You can write the variable by using double @ signs, e.g. \verb|@@q1_p|.
		Variables that are undefined are just printed in the tex file, e.g. @asdfgh 
		\begin{itemize}
			\item \verb|@@q1_p| is $@q1_p$
			\item \verb|@@q1_q| is $@q1_q$
			\item \verb|@@q1_c| is $@q1_c$
			\item \verb|@@q1_d| is $@q1_d$
		\end{itemize}

		Under the \#evaluation section, I used those variables to calculate

		\begin{itemize}
			\item $@q1_p \times @q1_q = @prod1$
			\item $@q1_p \times @q1_q + @q1_c = @prod2$
		\end{itemize}

		Formulas can be printed, but I haven't yet made them pretty:
		\[
			@formula
		\]
		
	\item
		A @q2_animal has @legs legs.
		
	\item 
		Lists are supported.
		The element at index @q3_index of @q3_elements is @q3_choice

		Fractions are assumed until a floating point number is introduced.

		\begin{align*}
			c &= @q3_c\\
			d &= @q3_d\\
			e &= @q3_e\\
			c\cdot d &= @q3_cd\\
			c\cdot e &= @q3_ce\\
			c\cdot d + e &= @q3_cde\\
			c\cdot d + e + 0.5 &= @q3_cde2\\
			c\cdot d + e + 0.5 &\approx @q3_cde2_round
		\end{align*}
		The last line uses the built-in \texttt{round} function.
		Only a handful of builtins are implemented.
			But you can add them easily; refer to the Environment class in syntaxtree.py.


		Functions can be defined and evaluated. 
		\begin{align*}
			f(3) &= @fe\\
			g(3) &= @ge\\
			h(3) &= @he
		\end{align*}
		"Void" is used when something is undefined or goes wrong.
		In this case, $h$ is a function that needs two variables:
		$h(1,2) = @he2$.
		
		Note that the variable @@fevals is a list of functions.
		Evaluating a list, e.g. \verb|[f,g,h](1)| passes that argument to each element of the list. 

		What happens if you try to evaluate something that isn't a function? 
		Undefined, because everything is considered a function.
		Numbers/strings/etc. are treated as functions of arbitrary arity. 
		E.g. since \verb|@@q3_c| contains the number $@q3_c$, we can try evaluating it:
		\begin{itemize}
			\item \verb|q3_c(1)| yields $@q3c1$
			\item \verb|q3_c(1,2)| yields $@q3c12$
			\item \verb|q3_c()| yields $@q3cev$
		\end{itemize}
		


	\item
		Strings are just strings.

		The string in \verb|@@q4_str| is @q4_str.

	\item 
		If the column is not in JSON format, it is imported as a string.

		Variable \verb|@@q5| contains the string ``@q5''

	\item
		Because Column J (index 9) is not included under \#colnames, it is not imported from the spreadsheet.
\end{enumerate}



\end{document}
